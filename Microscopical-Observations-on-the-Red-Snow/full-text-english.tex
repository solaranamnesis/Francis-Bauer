\documentclass[a4paper, 12pt, oneside]{article}
\usepackage[utf8]{inputenc}
\usepackage{fouriernc}
\usepackage{booktabs}
\setlength{\emergencystretch}{15pt}
\usepackage{fancyhdr}
\usepackage{graphicx}
\usepackage{hyperref}
\usepackage{csquotes}
\usepackage{microtype}
\graphicspath{ {./} }
\usepackage{float}
\begin{document}
\begin{titlepage} % Suppresses headers and footers on the title page
	\centering % Centre everything on the title page
	\scshape % Use small caps for all text on the title page

	%------------------------------------------------
	%	Title
	%------------------------------------------------
	
	\rule{\textwidth}{1.6pt}\vspace*{-\baselineskip}\vspace*{2pt} % Thick horizontal rule
	\rule{\textwidth}{0.4pt} % Thin horizontal rule
	
	\vspace{1.5\baselineskip} % Whitespace above the title
	
	{\LARGE Microscopical Observations\\ on the Red Snow }
	
	\vspace{1\baselineskip} % Whitespace above the title

	\rule{\textwidth}{0.4pt}\vspace*{-\baselineskip}\vspace{3.2pt} % Thin horizontal rule
	\rule{\textwidth}{1.6pt} % Thick horizontal rule
	
	\vspace{1\baselineskip} % Whitespace after the title block

	%------------------------------------------------
	%	Subtitle
	%------------------------------------------------
	
	{By Francis Bauer} % Subtitle or further description
	
	\vspace*{1\baselineskip} % Whitespace under the subtitle
	
	{Extracted from the Quarterly Journal, No. 14.} % Subtitle or further description
	
	\vspace*{1\baselineskip} % Whitespace under the subtitle
	%------------------------------------------------
	%	Subtitle
	%------------------------------------------------
	
	%------------------------------------------------
	%	Editor(s)
	%------------------------------------------------
    \vspace*{\fill}

	{\small\scshape }

    July 1819
	
	{\small Attribution NonCommercial ShareAlike 4.0 International } % Publisher
\end{titlepage}
\setlength{\parskip}{1mm plus1mm minus1mm}
\clearpage
\section{Francis Bauer, F. L. S. F. II. S., in a Letter to W. T. Brande, Esq, \emph{Sec.} R. S., \emph{etc.}}
\paragraph{}
Dear Sir,

Although I had heard various reports and opinions respecting that curious phenomenon, the Red Snow (discovered during the late Northern Expedition, under the command of Captain Ross, on the 17$^{th}$ of August, 1818, in Baffin's Bay, in lat. 75° 54$^{\prime}$, N. and long. 67° 15$^{\prime}$, W.) I had no opportunity of seeing it, until the 28$^{th}$ of February last, when I received a quart bottle full of the melted snow, for the purpose of ascertaining (as far as it could be effected by microscopical observation,) whether the red colouring matter were an animal or a vegetable substance.

I now venture to lay before you a detailed account of the mode and result of my investigation; and if you should deem it deserving the honour of insertion in the Quarterly Journal of the Royal Institution, it is entirely at your service.

After the bottle had remained at rest for eighteen hours, I found on examination that is contents consisted of perfectly clear water, having deposited a sediment at the bottom of the bottle, not quite a quarter of an inch in thickness (or in proportion to the water, as 1 to 34.) apparently consisting of an extremely fine powder, of a dark red colour.

Carefully uncorking the bottle, without disturbing its contents, I dipped a small ivory instrument into the clear water, and placed a drop, which covered a space of about a sixteenth part of a square inch, upon a plain glass; and bringing it under the microscope, I observed it to be pure water, on the surface of which were floating about 15 or 20 extremely minute organized spherical corpuscles, or globules, of different sizes, perfectly colourless and transparent. I repeated these operations several times, and found always the same appearances.

I then shook the bottle, in order to mix the sediment with the water, which was very soon effected, and the whole contents became tinged with a light crimson colour. Putting a small drop of this coloured water upon the plain glass, and bringing it within the field of the microscope, I observed some hundreds of similar globules, of various sizes, most of them nearly opaque, and of a fine dark red colour, all of which soon sunk to the bottom, but the transparent and colourless globules remained floating on the surface of the water.

These globules I could compare to nothing but either the pollen of some plants, or to the minute fungi of the genus \emph{uredo}.

With this impression on my mind, I examined the subject more closely, and, employing stronger magnifying powers, soon found several individuals still adhering to their pedicles, the same as I have found in most species of \emph{uredo}, and which distinguishes these minute fungi from the pollen of some plants.

As the water on the glass under the microscope gradually evaporated, I observed also the same granulated glutinous substance which always issues from those fungi when ripe, and which I believe contains its seed. When the water is entirely evaporated, and the globules become quite dry, they stick together in the same manner as all species of \emph{uredo} do, and in that state they can hardly be distinguished in shape or colour from the \emph{uredo faetida}.

Having dried a sufficient quantity of these globules, I put them upon some hot iron, and the smell of the fumes proved also that they are vegetable matter.

The above experiments I repeated, with a sufficient quantity of the \emph{uredo faetida}, and the results were precisely the same; the ripe and coloured fungi sink to the bottom of the glass, or bottle, and form the same kind of sediment, and the unripe and colourless fungi remain floating; and when put, in a dry state, upon hot iron, the smell of the fumes is the same as that which is produced by the red globules.

It was at this stage of my investigation, and after I had ascertained the above facts, that I gave it as my opinion, that the substance which gives the red colour to the snow is not of animal origin, but a fungus of the genus \emph{uredo}. This was on the first of March last, several weeks previous to the publication of Captain Ross's narrative of the voyage, and before I knew that any one else had investigated the subject, which I only learned when I read the accounts given in the publication; but at that time I had already pursued the subject further, and ascertained additional facts, all of which tended to confirm my first opinion; at the time when I expressed that opinion, I had only seen the fungi in a detached and loose state (see Figs. 1, 2, 3, and 4, in the annexed Plate,) but on the 14$^{th}$ of March, as I was pouring out a larger quantity of the contents of the bottle, I observed several flakes, of a jelly-like white substance, with many of the full-grown red fungi adhering to it, which, when brought under the microscope, proved to be the same cellular or articulate root, or spawn, which is common to most species of \emph{uredo}, (see Figs. 5, 6, 7, 8, and 10.)

On the 16$^{th}$ of March I poured a considerable quantity of the coloured water into a large conical-shaped rummer, to obtain more of the sediments; when, after it had been standing at rest twenty-four hours, I found, on examination, that, although a considerable quantity of sediment had been formed, the inner surface of the glass, as far as the water reached, was entirely covered with a single layer of the red fungi. This appearance continued unaltered till the fourth day, when I observed the fungi were gradually losing their colour, and small flakes of the jelly-like spawn became evident in several spots on the inner suface of the glass; and after the glass had stood at rest three days longer, I found the fungi had entirely lost their colour, and the new-formed spawn had increased considerably; and on bringing a small portion of the substance within the field of the microscope, I found the white spawn the same in appearance as that which I found in the original bottle, and a great many very minute colourless young fungi were adhering to its surface.

After the glass had stood at rest another week, I examined another portion, and found the new spawn had not only considerably increased in quantity, but even the mark made on the glass where I scraped the first portion off, was nearly obliterated, being overgrown by newly-formed spawn, and the new fungi had nearly attained the ordinary size of full-grown fungi, but were still perfectly colourless, (see Fig. 6.)

I have since repeatedly examined the contents of the glass, but have not observed any material change. The increase of new fungi continued evident for about three weeks; since that time the marks on the glass from which the substance had occasionally been removed, remained visible and uncovered, and the fungi, accumulating in large clusters, detached themselves from the glass, and sunk to the bottom, without ever attaining the red colour, though the substance has now been exposed to the open air for the last ten days and nights. Hence it appears evident, that the new-formed fungi never come to perfect maturity; and that, when the seed of the primitive plant is exhausted, the increase ceases.

The original red, as well as the newly-produced colourless fungi, when left to dry, become both the same brownish gray colour, see Fig. 7;) but if the red fungi, whilst fresh, are bruised and rubbed on the skin of the hand or the face, they form a pigment of the most bright vermilion, or red lead colour, which remains unchanged, even over night, and till it is washed off with soap and water.

The results of Dr. Wollaston's chemical analysis, published in Captain Ross's narrative, correspond also in every material point with those obtained by Tessier, in experiments made with \emph{uredo faetida}, and \emph{uredo segetum}. (See \emph{Traite des Maladies des Grains}. Par M. l'Abbe Tessier, p. 225-235.)

Having diligently ascertained the above facts, I feel now not the slightest hesitation in saying, that the substance which gives the red appearance to the snow is a new species of \emph{uredo}, and for which I think the most proper specific name is \emph{nivalis}.

There can be no doubt but this new species of \emph{uredo} grows upon the snow where it is found; for it appears impossible that the substance could be brought from any distance by the wind, or any other means, to that spot in so great a quantity; for it is stated in Captain Ross's narrative, that the extent of the crimson cliffs is about eight miles; the sides of the hills on which it was found are about six hundred feet high; and he says, the party which he had sent to procure some of the snow ``found that the snow was penetrated even down to the rock, in many places to a depth of ten or twelve feet, by the colouring matter; and that it had the appearance of having been a long time in that state.'' But it is not stated in how many places the party had been digging ten or twelve feet deep in the snow.

In a journal of the same voyage, published by an officer of the Alexander; in the account given there of the red snow, it is stated, p. 63. ``This substance, whatever it may be, is very plentiful, on this part of the coast, the snow being covered with it in different places to a considerable extent. It is soluble in water, to which it gives a deep red colour, but, when allowed to settle a little, sinks to the bottom, leaving the water almost colourless. It is worthy of remark, that this colouring matter, be it what it may, \emph{does not penetrate more than an inch or two} beneath the surface of the snow,'' \emph{etc.}, \emph{etc.} This appears certainly more probable; however, it will be better to leave this little difference to be settled by the respective travellers.

Mr. Brown, in a very short note, published in Captain Ross's narrative, expresses an opinion that the plant might be a \emph{tremella}; and quotes \emph{tremella cruenta}, in the English Botany, Fig. 1800. That plant I have not yet seen in nature, but, judging from the quoted figure, as well as from the description, I am firmly persuaded it is no \emph{tremella}, but most likely an \emph{uredo}. The authors of the English Botany, in their description of this plant, say, ``When examined with a microscope, it proves to be a congerics of extremely minute, pellucid, globular granulations, all equal in size.''

They conclude their description with saying, ``We are well aware that it can only rank as a \emph{tremella} till more observations are made on the subject.''

It is true all the species of \emph{uredo} which I have hitherto examined, and those described by Persoon, are parasitical plants, growing upon other vegetables; but that I think does not prove that there are none that grow otherwise. I know at least one instance that these same parasitical plants can, and do, vegetate and propagate on other substances than living plants, for, in 1807, during my investigations of the diseases in corn, I put some barley and oat ears, which partially infected with smut (which is the \emph{uredo segetum},) into brown paper, to preserve as specimens, in which, on examining them in about three or four months afterwards, I found the fungi had not only entirely consumed several of the ears, but had continued growing and multiplying on the paper, on which they had spread from the several spikets of the ears in distinct rays, of two or three inches in length, and the quantity of fungi thus produced on the paper is at least three times the quantity the original ears could have contained; these specimens are still in my possession. I have no doubt but that the \emph{uredo segetum}, as well as the \emph{uredo faetida}, vegetate and propagate in like manner on the soil, it being a known fact that the purest seed-corn, sown on land which, several years before, produced those diseases, will likewise be infected, though, during the interval, no wheat or barley has been cultivated on that ground; and it is not probable that these fungi and their seed should have been lying in the ground inactive for several years; but their extreme minuteness, and their dark colour, renders it almost impossible to detect them on the ground; perhaps some future observations might ascertain these facts more decisively.

The method I have used to ascertain the real shape and size of these fungi, is that which a thirty year's practice and experience has convinced me to be the most simple and most accurate; and, in cases of such extremely minute objects, the only practicable mode, \emph{viz.}, by means of a glass micrometer, in a Dollond's compound microscope.

The micrometer employed on this occasion, was a line of an inch in length, divided into 400 parts, which divides the square inch into 160,000 parts, or squares. On examination under the microscope, I found that four full-grown fungi, in a close line, occupy exactly the space of a $\frac{1}{400}$ part of an inch in length, (see Fig. 1;) therefore 1,600 of these fungi are required to form a whole inch in length. Hence the real diameter of an individual full-grown fungus of \emph{uredo nivalis} is the \emph{one thousand six hundredth part of an inch}.

In the annexed plate, Fig. 1, represents a $\frac{1}{160,000}$ part of a square inch, and shows that, in order to cover its whole surface, 16 fungi are required; consequently, to cover the surface of an entire square inch, \emph{two million five hundre and sixty thousand} such fungi are requisite.

The above $\frac{1}{160,000}$ part of a square inch (Fig. 1.) being represented of the size of a whole square inch; it is therefore magnified 400 times in diameter, and 160,000 times superficies, and every object represented within that square is consequently magnified in the same degree, \emph{viz.}, \emph{four hundred times} in diameter, and \emph{one hundred and sixty thousand times} in superficies.

The form and size of the fungi is ascertained, as above stated, by means of the glass micrometer, in a transparent state, and their colour, by placing them on white paper, and viewing them with a strong magnifying single lens, in an opaque state.

To afford an opportunity for immediate comparison, I have represented in the annexed plate, two well-known species of \emph{uredo}, \emph{viz.} \emph{uredo faetida}, which grows within the grains of wheat, and constitutes the disease known by the name of \emph{smut-balls}, or \emph{pepper-brand} (see Figs. 10, 11,) and the \emph{uredo graminis}, which grows on the leaves of wheat and most grasses, and occasions the disease called the \emph{red rust}, (Fig. 8, 9.) Those figures are magnified in the same degree as \emph{uredo nivalis}.

Francis Bauer.

Kew-Green, 25$^{th}$ April, 1819.
\clearpage
\section{Explanation of the Plate. (Plate VI.)}
\paragraph{}
Fig. 1. A $\frac{1}{160,000}$ part of a square inch, containing 16 full-grown fungi of \emph{uredo nivalis}, magnified 400 times in diameter, or 160,000 times in superficies.

Fig. 2. Some young fungi.

Fig. 3. Some full-grown fungi, but larger than the usual, or general, size.

Fig. 4. A large sized fungus, in a colourless and transparent state, when it is seen that these fungi are neither recticular nor cellular, but the granular contents are in that state perceptible.

Fig. 5. A cluster of fungi of different sizes, on their spawn, as found in the original bottle.

Fig. 6. A cluster of nearly full-grown colourless fungi, on their spawn, as lately grown in an open glass in the house.

Fig. 7. A small cluster of full-grown fungi, in a dry state. All the above figures represent \emph{uredo nivalis}.

Fig. 8. A cluster of the same fungi on their spawn.

Fig. 9. A $\frac{1}{400}$ part of an inch in diameter, sustaining 3 full-grown fungi of \emph{uredo graminis}. In these fungi a hexagonal recticulation is perceptible.

Fig. 10. A cluster of the same fungi of different sizes and ages, on their spawn.

Fig. 11. A $\frac{1}{400}$ part of an inch in diameter, sustaining 4 full-grown fungi of \emph{uredo faetida}.

Every object in this plate is magnified \emph{four hundred} times in diameter, and \emph{one hundred and sixty thousand} times in superficies; and if the objects were contemplated as solids, they would be magnified \emph{sixty-four millions of times}.
\clearpage

\begin{figure}[b]
\includegraphics[width=1.12\textwidth,keepaspectratio]{Francis-Bauer-Plate-All-Figures-rotated-smaller.png}
\centering
\end{figure}
\clearpage
\end{document}
